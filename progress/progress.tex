\documentclass[]{scrartcl}

\title{\LARGE \color{darkgreen}Radio K.A.O.S\color{black}:
              Cognitive Radio with Dynamic Spectrum Sharing Engine}
\subtitle{\vspace{3ex} \Large Project Progress Report}
\author{
    \large \textit{Mostafa Abd El-Aziz} (67)\\
}
\date{\large \today}

% font
\usepackage[T1]{fontenc}
\usepackage{lmodern}
\usepackage{sfmath}
\usepackage{microtype}
\usepackage[utf8]{inputenc}
% \usepackage{cfr-lm} % font package

% listings
\usepackage{listings}
\usepackage{color}
\definecolor{mygreen}{rgb}{0,0.6,0}
\definecolor{mygray}{rgb}{0.5,0.5,0.5}
\definecolor{mymauve}{rgb}{0.58,0,0.82}
\lstset{ %
  backgroundcolor=\color{white},
  basicstyle=\footnotesize\ttfamily,
  breakatwhitespace=false,
  breaklines=true,
  captionpos=b,
  commentstyle=\color{mygreen},
  deletekeywords={...},
  escapeinside={\%*}{*)},
  extendedchars=true,
  frame=single,
  keepspaces=true,
  keywordstyle=\color{blue},
  language=r,
  otherkeywords={*,...},
  numbers=none,
  numbersep=5pt,
  numberstyle=\tiny\color{mygray},
  rulecolor=\color{black},
  showspaces=false,
  showstringspaces=false,
  showtabs=false,
  stepnumber=2,
  stringstyle=\color{mymauve},
  tabsize=2,
  title=\lstname
}

% margins
\usepackage[margin=3cm,a4paper]{geometry}

% figures and tables
\usepackage{graphicx}
\usepackage{float}

% table of contents
\usepackage{tocloft}
\renewcommand{\cftsecleader}{\cftdotfill{\cftdotsep}} % for chapters
\renewcommand{\cftpartleader}{\cftdotfill{\cftdotsep}} % for parts

% section style
\usepackage{sectsty}
\sectionfont{\LARGE}

% page orientation
\usepackage{lscape}

% hyperlinks
\usepackage{hyperref}

% colors
\definecolor{darkgreen}{RGB}{0, 100, 0}

%------------------------------------------------------------------------------%

\begin{document}

% university logo
\titlehead{

    \small Alexandria University\\
    \small Faculty of Engineering\\
    \small Computer and Systems Engineering Department\\
    \small CS 436 - Topics in Computer Networks

    \vspace{-2.2cm}
    \hfill
    \includegraphics[scale=0.6]{logo.png}
}

% document title
\maketitle

%------------------------------------------------------------------------------%

\section{GNU Radio}

\begin{itemize}

\item Finished introductory tutorial on
\newline \url
{https://gnuradio.org/redmine/projects/gnuradio/wiki/Guided\_Tutorials}
\newline
and learned how to use GNU Radio Companion.

\item Implemented a couple of flow graphs to get used to the companion.

\end{itemize}

\section{Readings}

\begin {itemize}

\item T. Charles Clancy, Zhu Ji, Beibei Wang, K. J. Ray Liu.
      ``\textit{Planning Approach
      to Dynamic Spectrum Access in Cognitive Radio Networks}''\\
      \textbf{Summary}:
      In this paper, an implementation for a general-purpose cognitive radio
      engine is proposed. The CR engine shall consist of a knowledge base,
      reasoning engine, and learning engine:
      \begin{itemize}
        \item Knowledge base: stores the semantics of dynamic spectrum access
              in action description language (ADL) as a set of facts and rules
              (actions).
        \item Learning engine: manipulates the knowledge base from experience.
              Learning engine may only be run to train a newly initialized
              radio, or it could be run periodically as the radio operates.
        \item Reasoning engine: executes the actions stored in the knowledge
              base. It keeps evaluating the objective function over
              all possible choices (all possible available bands and
              center frequencies). The goal of the objective function
              is to maximize capacity. A Primary-Prioritized Markov
              Approach (PPMA) for dynamic spectrum access is proposed,
              which is based on a fairly simple 8-state finite state machine.
              The interactions between the primary user (holder of licensed
              band) and the secondary users (unlicensed users) are modeled as
              a Primary-Prioritized Continuous-Time Markov Chain (PPCTMC).
      \end{itemize}

\item Usama Mir,
      Leila Merghem-Boulahia,
      Moez Esseghir,
      D. Gaiti.
      ``\textit{Dynamic spectrum sharing for cognitive radio networks
      using multiagent system}''\\
      \textbf{Summary}:
      The discusses the details of design of architecture for an ASIP
      (Application Specific Instruction Processor) for SDR (Software-Defined
      Radio) with a single bus and dedicated paths. The paper proposes
      an instruction set with the following instruction groups:
      \begin{enumerate}
        \item Arithmetic copy/move instructions
        \item Integer arithmetic store with constant
        \item Floating-point Arithmetic store with constant
        \item Integer branch instructions
        \item Floating-point branch instructions
        \item Base-band and channel coding instructions
      \end{enumerate}
      Channel coding, decoding, modulation, and demodulation modules
      are all embedded in the ASIP.

\item Thomas Charles Clancy III,
      ``\textit{Dynamic Spectrum Access in Cognitive Radio Networks}'',
      Doctor of Philosophy thesis.\\
      \textbf{Summary}:
      I read Chapter 2 which introduces cognitive radio, explains
      information theory basics behind cognitive radio networks,
      and finally, shows various multiple access schemes for multiplexing
      users' communications.

\item Upamanyu Madhow, ``\textit{Introduction to Communication Systems}''\\
      \textbf{Summary}:
      This is a textbook that articulates a selection of concepts
      that the author deems fundamental to communication system design.
      I read parts of chapter 3 [Analog Communications], and parts of
      chapter 4 [Digital Modulation].

\end{itemize}

\section{FPGA}

\begin{itemize}

\item Implemented VHDL code to transmit music over FM using FPGA. The code
      was tested using Spartan-3 FPGA and a radio receiver. This exercise
      gave us an idea about how FPGAs could be used to modulate and
      send radio waves.

\end{itemize}

\section{USRPs}

\begin{itemize}

\item On 14 April we have registered an account on SmartCI's Cognitive
      Radio Cloud (CRC). The CRC currently provides users with up to
      5 USRPs, 8 wifi interfaces, and 8 RTLs.

\item I learned how to use two of the USRPs to exhange
      signals: one could send a signal on a carrier frequency, and
      the other would sense the spectrum and detect a signal on the carrier.
      The 2 USRPs used where placed in the meeting room @ VTMENA second floor.

\end{itemize}

\end{document}
